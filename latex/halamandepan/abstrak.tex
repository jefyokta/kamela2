%-----------------------------------------------------------------------------------------------%
%
% % Oktober 2022
% Template Latex untuk Laporan Kerja Praktek Program Studi Sistem informasi ini
% Dikembangkan oleh Daffa Takratama Savra (daffatakratama13@gmail.com)

% Template ini dikembangkan dari template yang dibuat oleh Inggih Permana (inggihjava@gmail.com).

% Orang yang cerdas adalah orang yang paling banyak mengingat kematian.
%
%-----------------------------------------------------------------------------------------------%
\fontsize{12}{14.4}
\begin{center}\MakeUppercase{\textbf{Abstrak}}\end{center}

\noindent
\fontsize{10pt}{12pt}\selectfont
Pengelolaan pemesanan rumah secara manual seringkali menimbulkan berbagai kendala, seperti pencatatan yang tidak konsisten, kehilangan data, serta ketidakakuratan informasi yang dapat memengaruhi kepuasan pelanggan. Di Perumahan Kamela Permai, proses pemesanan masih dilakukan dengan metode tradisional, yang memperlambat pengolahan data dan pengambilan keputusan. Untuk mengatasi masalah ini, dikembangkanlah “Sistem Informasi Pemesanan Rumah” yang bertujuan untuk meningkatkan efisiensi dan jangkauan dalam proses pemesanan. Sistem ini dirancang menggunakan metode Extreme Programming (XP) dengan pendekatan OOAD (Object Oriented Analysis and Design) untuk memastikan pengembangan perangkat lunak yang adaptif dan kolaboratif. Dengan sistem ini, pelanggan dapat melakukan pemesanan secara online, mengecek status unit secara real-time, dan memperoleh informasi lengkap mengenai ketersediaan rumah. Di sisi lain, pengelola perumahan dapat dengan mudah mengakses data pemesanan, memantau stok unit, dan membuat laporan. Diharapkan, sistem ini mampu meningkatkan efektivitas dan efisiensi pengelolaan pemesanan rumah di Perumahan Kamela Permai serta memberikan pengalaman yang lebih baik bagi pelanggan dan pengelola.\\
\noindent{\textbf{Kata Kunci:} Sistem Informasi, Pemesanan Perumahan, Berbasis Web, \textit{OOAD}, \textit{Extreme Programming} \\