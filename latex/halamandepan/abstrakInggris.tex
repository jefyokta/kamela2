%-----------------------------------------------------------------------------------------------%
%
% % Oktober 2022
% Template Latex untuk Laporan Kerja Praktek Program Studi Sistem informasi ini
% Dikembangkan oleh Daffa Takratama Savra (daffatakratama13@gmail.com)

% Template ini dikembangkan dari template yang dibuat oleh Inggih Permana (inggihjava@gmail.com).

% Orang yang cerdas adalah orang yang paling banyak mengingat kematian.
%
%-----------------------------------------------------------------------------------------------%
\fontsize{12}{14.4}
\begin{center}\MakeUppercase{\textbf{\emph{Abstract}}}\end{center}

\noindent
\fontsize{10pt}{12pt}\selectfont
\emph{The manual management of house reservations often faces various challenges, such as inconsistent record-keeping, data loss, and inaccurate information, which can affect customer satisfaction. At Kamela Permai Residence, the reservation process is still carried out traditionally, slowing down data processing and decision-making. To address these issues, a House Reservation Information System was developed to improve the efficiency and reach of the reservation process. The system was designed using the Extreme Programming (XP) method with an Object-Oriented Analysis and Design (OOAD) approach to ensure an adaptive and collaborative software development process. This system allows customers to make reservations online, check unit availability in real-time, and access detailed information about available houses. On the management side, administrators can easily access reservation data, monitor unit stocks, and generate reports. This system is expected to improve the effectiveness and efficiency of the reservation management process at Kamela Permai Housing while providing a better experience for both customers and administrators.}\\
\noindent{\emph{\textbf{Keywords:} \textit{Information System, House Reservation, Web-Based, OOAD, Extreme Programming.}}} \\