%-----------------------------------------------------------------------------------------------%
%
% % Oktober 2022
% Template Latex untuk Laporan Kerja Praktek Program Studi Sistem informasi ini
% Dikembangkan oleh Daffa Takratama Savra (daffatakratama13@gmail.com)

% Template ini dikembangkan dari template yang dibuat oleh Inggih Permana (inggihjava@gmail.com).

% Orang yang cerdas adalah orang yang paling banyak mengingat kematian.
%
%-----------------------------------------------------------------------------------------------%

%-----------------------------------------------------------------------------%
\chapter{\babLima}
% -----------------------------------------------------------------------------%
\section{Kesimpulan}
% -----------------------------------------------------------------------------%
\par Berdasarkan hasil kerja praktek penulis yang berjudul Rancang Bangun Sistem Informasi Absensi Pegawai di SMPN 3 Bangkinang maka dapat diambil kesimpulan sebagai berikut:
\begin{enumerate}
\item Analisa sistem yang sedang berjalan pada SMPN 3 Bangkinang, memiliki beberapa permasalahan dan hambatan dalam absensi pegawainya. Maka dengan adanya analisa usulan baru yaitu Sistem Informasi Absensi Pegawai Berbasis Web akan menghasilkan solusi serta gambaran sistem yang lebih baik dan absensi pegawai dapat dilakukan secara cepat dan efisien.
\item Dengan dibangunnya sistem informasi absensi pegawai ini dapat mempermudah jalannya absensi kepegawaian pada SMPN 3 Bangkinang, tanpa adanya kecurangan dan keterlambatan para pegawai.
\item Mengurangi biaya operasional dan memberikan kemudahan bagi pegawai untuk melakukan absensi tanpa harus melakukan tanda tangan secara tertulis diatas kertas.
\end{enumerate}
% -----------------------------------------------------------------------------%
\section{Saran}
\par Saran yang membangun untuk kemajuan penelitian ini yaitu untuk penelitian selanjutnya diharapkan agar bisa membangun sebuah sistem yang telah dihosting agar pihak instansi lebih mudah untuk menggunakannya, karena pada sistem yang diusulkan sekarang ini sistem yang diberikan belum dihosting. Dan juga diharapkan agar bisa membangun sistem yang tampilannya lebih menarik dan mudah dimengerti. Semoga Sistem Informasi Absensi Pegawai ini pada tahap selanjutnya dapat dilakukan pengembangan dan implementasi menjadi sistem yang lebih baik dan sempurna terhadap Sistem Informasi Absensi Pegawai.
% -----------------------------------------------------------------------------%